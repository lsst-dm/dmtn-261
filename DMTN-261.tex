\documentclass[DM,authoryear]{lsstdoc}
\input{meta}

% Package imports go here.

% Local commands go here.

\title{On the use of the CSDC ANTARES Broker for the Rubin/LSST Alert Filtering Service.}

% Memo on the provision of the Rubin/LSST Alert Filtering System by ANTARES
\author{V. Krabbendam, T. Matheson, R. Blum}

\setDocRef{DMTN-261}
\setDocUpstreamLocation{\url{https://github.com/lsst-dm/dmtn-261}}

\date{2024-06-21}

\begin{document}

\setMemoto{ L. Guy, E. Bellm, P. Marshall}
        
\mkmemotitle

During steady-state operations, Rubin Observatory's LSST will produce about ten million world public alerts of transients, variable, and moving objects. Science users will work with community alert
brokers to crossmatch, filter, and classify these alerts in order to identify the subset that require real-time followup observations. 
At the time the vision for this ambitious and comprehensive alert system was developed, no community alert brokers were functional.

Accordingly, the Rubin project provided a fallback: a simple “alert filtering system” (AFS). 
The AFS would allow Rubin data-rights holders to apply simple user-defined or predefined filters to the Rubin alert stream. 
This mitigated the risk that community alert brokers would not be available, or not provide the functionality required to fulfill users' needs.

Today, the situation is quite different. 
Seven community alert brokers have been approved for direct access to the full Rubin alert stream, with two more planning to operate downstream of a full-stream broker. 
Several brokers have functionality which mirrors that planned for the Rubin AFS. 
Nevertheless it is reasonable for the Rubin project to ensure that Rubin data-rights holders have access to AFS-like capabilities throughout the survey, even if that service is provided by another entity.

Therefore Rubin Observatory commits to provide resources to enable the use of the Arizona-NOIRLab Temporal Analysis and Response to Events System (ANTARES),   \citep{2021AJ....161..107M} AFS for Rubin. 
Rubin will work with CSDC to develop a detailed plan to commission the AFS following the technical description given in the  technical note \citeds{dmtn-226}. 
It is expected the AFS will be commissioned along with the rest of the Rubin system. 

\clearpage
\appendix
% Include all the relevant bib files.
% https://lsst-texmf.lsst.io/lsstdoc.html#bibliographies
\section{References} \label{sec:bib}
\renewcommand{\refname}{} % Suppress default Bibliography section
\bibliography{local,lsst,lsst-dm,refs_ads,refs,books}

% Make sure lsst-texmf/bin/generateAcronyms.py is in your path
\section{Acronyms} \label{sec:acronyms}
\addtocounter{table}{-1}
\begin{longtable}{p{0.145\textwidth}p{0.8\textwidth}}\hline
\textbf{Acronym} & \textbf{Description}  \\\hline

DM & Data Management \\\hline
\end{longtable}

% If you want glossary uncomment below -- comment out the two lines above
%\printglossaries




\end{document}